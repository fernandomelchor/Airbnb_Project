
%% bare_conf.tex
%% V1.4b
%% 2015/08/26
%% by Michael Shell
%% See:
%% http://www.michaelshell.org/
%% for current contact information.
%%
%% This is a skeleton file demonstrating the use of IEEEtran.cls
%% (requires IEEEtran.cls version 1.8b or later) with an IEEE
%% conference paper.
%%
%% Support sites:
%% http://www.michaelshell.org/tex/ieeetran/
%% http://www.ctan.org/pkg/ieeetran
%% and
%% http://www.ieee.org/

%%*************************************************************************
%% Legal Notice:
%% This code is offered as-is without any warranty either expressed or
%% implied; without even the implied warranty of MERCHANTABILITY or
%% FITNESS FOR A PARTICULAR PURPOSE!
%% User assumes all risk.
%% In no event shall the IEEE or any contributor to this code be liable for
%% any damages or losses, including, but not limited to, incidental,
%% consequential, or any other damages, resulting from the use or misuse
%% of any information contained here.
%%
%% All comments are the opinions of their respective authors and are not
%% necessarily endorsed by the IEEE.
%%
%% This work is distributed under the LaTeX Project Public License (LPPL)
%% ( http://www.latex-project.org/ ) version 1.3, and may be freely used,
%% distributed and modified. A copy of the LPPL, version 1.3, is included
%% in the base LaTeX documentation of all distributions of LaTeX released
%% 2003/12/01 or later.
%% Retain all contribution notices and credits.
%% ** Modified files should be clearly indicated as such, including  **
%% ** renaming them and changing author support contact information. **
%%*************************************************************************


% *** Authors should verify (and, if needed, correct) their LaTeX system  ***
% *** with the testflow diagnostic prior to trusting their LaTeX platform ***
% *** with production work. The IEEE's font choices and paper sizes can   ***
% *** trigger bugs that do not appear when using other class files.       ***                          ***
% The testflow support page is at:
% http://www.michaelshell.org/tex/testflow/

\documentclass[conference]{IEEEtran}

% *** CITATION PACKAGES ***
%
%\usepackage{cite}
% cite.sty was written by Donald Arseneau
% V1.6 and later of IEEEtran pre-defines the format of the cite.sty package
% \cite{} output to follow that of the IEEE. Loading the cite package will
% result in citation numbers being automatically sorted and properly
% "compressed/ranged". e.g., [1], [9], [2], [7], [5], [6] without using
% cite.sty will become [1], [2], [5]--[7], [9] using cite.sty. cite.sty's
% \cite will automatically add leading space, if needed. Use cite.sty's
% noadjust option (cite.sty V3.8 and later) if you want to turn this off
% such as if a citation ever needs to be enclosed in parenthesis.
% cite.sty is already installed on most LaTeX systems. Be sure and use
% version 5.0 (2009-03-20) and later if using hyperref.sty.
% The latest version can be obtained at:
% http://www.ctan.org/pkg/cite
% The documentation is contained in the cite.sty file itself.

% *** GRAPHICS RELATED PACKAGES ***
%
\ifCLASSINFOpdf
  \usepackage[pdftex]{graphicx}
  \graphicspath{{img/}}
  \DeclareGraphicsExtensions{.pdf,.jpeg,.png}
\else
  % or other class option (dvipsone, dvipdf, if not using dvips). graphicx
  % will default to the driver specified in the system graphics.cfg if no
  % driver is specified.
  % \usepackage[dvips]{graphicx}
  % declare the path(s) where your graphic files are
  % \graphicspath{{../eps/}}
  % and their extensions so you won't have to specify these with
  % every instance of \includegraphics
  % \DeclareGraphicsExtensions{.eps}
\fi
% graphicx was written by David Carlisle and Sebastian Rahtz. It is
% required if you want graphics, photos, etc. graphicx.sty is already
% installed on most LaTeX systems. The latest version and documentation
% can be obtained at:
% http://www.ctan.org/pkg/graphicx
% Another good source of documentation is "Using Imported Graphics in
% LaTeX2e" by Keith Reckdahl which can be found at:
% http://www.ctan.org/pkg/epslatex
%
% latex, and pdflatex in dvi mode, support graphics in encapsulated
% postscript (.eps) format. pdflatex in pdf mode supports graphics
% in .pdf, .jpeg, .png and .mps (metapost) formats. Users should ensure
% that all non-photo figures use a vector format (.eps, .pdf, .mps) and
% not a bitmapped formats (.jpeg, .png). The IEEE frowns on bitmapped formats
% which can result in "jaggedy"/blurry rendering of lines and letters as
% well as large increases in file sizes.
%
% You can find documentation about the pdfTeX application at:
% http://www.tug.org/applications/pdftex


% *** ALIGNMENT PACKAGES ***
%
%\usepackage{array}
% Frank Mittelbach's and David Carlisle's array.sty patches and improves
% the standard LaTeX2e array and tabular environments to provide better
% appearance and additional user controls. As the default LaTeX2e table
% generation code is lacking to the point of almost being broken with
% respect to the quality of the end results, all users are strongly
% advised to use an enhanced (at the very least that provided by array.sty)
% set of table tools. array.sty is already installed on most systems. The
% latest version and documentation can be obtained at:
% http://www.ctan.org/pkg/array

% correct bad hyphenation here
\hyphenation{op-tical net-works semi-conduc-tor}

\begin{document}
%
% paper title
% Titles are generally capitalized except for words such as a, an, and, as,
% at, but, by, for, in, nor, of, on, or, the, to and up, which are usually
% not capitalized unless they are the first or last word of the title.
% Linebreaks \\ can be used within to get better formatting as desired.
% Do not put math or special symbols in the title.
\title{Classifying Airbnb Listings in New York City}


% author names and affiliations
% use a multiple column layout for up to three different
% affiliations
\author{\IEEEauthorblockN{Jonathan Pichot, Fernando Melchor, and Avikal Somvanshi}
\IEEEauthorblockA{Center for Urban Science + Progress\\
New York University\\
New York, NY}
}

% conference papers do not typically use \thanks and this command
% is locked out in conference mode. If really needed, such as for
% the acknowledgment of grants, issue a \IEEEoverridecommandlockouts
% after \documentclass

% use for special paper notices
%\IEEEspecialpapernotice{(Invited Paper)}

% make the title area
\maketitle

% As a general rule, do not put math, special symbols or citations
% in the abstract
\begin{abstract}

\end{abstract}

\section{Introduction}
\IEEEPARstart
New York City is one of America's most visited cities, with iconic monuments like the
Statue of Liberty and the Empire State Building, a vibrant arts and cultural scene,
and an enviable conentration of bars and restaurants catering to every taste and budget.
It's a vast city including hundreds of neighborhoods each with their unique
history, economy, and character. The city attracted nearly 60 million visitors in 2015,
with a year over year growth of about 2 million vistors a year since 2010.\cite{nyc_and_company_nyc_nodate} Of course,
all these visitors need a place to stay.

Airbnb provides a platform for residents to rent space in their homes and apartments.
Founded in 2008, Airbnb's mission is to help people "monetize their extra space." They've
been very successful, with over 3 million listings in over 65,000 cities worldwide.\cite{airbnb_about_us}

\section{Literature Review}



\section{Data and Methods}
The four datasets collected for the analysis include:

% \begin{itemize}
%   \item Asthma Hospitilization Discharge Rate\cite{noauthor_information_nodate}
%   \item Building Energy Usage\cite{noauthor_gbee_nodate}
%   \item Neighborhood Income\cite{bureau_american_nodate}
%   \item Neighborhood Population\cite{bureau_american_nodate}
% \end{itemize}

\subsection{Asthma Hospitilization Discharge Rate}
Asthma hospitilization data was collected from New York State's Department of Health (NYDOH).
The NYDOH publishes hospital discharge numbers collected by the Statewide Planning
and Research Cooperative System (SPARCS). For this dataset, they focus only on hospital
discharges where the principal diagnosis was asthma. This dataset is a three-year average
over the years 2012 to 2014. The discharge rate per zip code is the total number of discharges
divided by the average population of that zip code over the same three-year span. Certain
zip codes with particularly low discharge rates are redacted for privacy reasons.

NYDOH publishes the data on their website as tables per county but they do not provide
a direct download. The data was collected by scraping the NYDOH website for
the five counties in New York City: Kings, Richmond, New York, Bronx, and Queens. Zip
codes without enough discharges to be published because of privacy and those with
such a low count to have a Relative Standard Error of greater than 30\% were also dropped.
This resulted in a total of 173 zip codes of asthma data.

\subsection{Building Energy Usage}
Local Law 84 in New York City requires owners of large buildings (and soon to be mid-sized
buildings) to benchmark the energy usage of their properties and submit this data
in a consistent way to the city. These benchmarks have come to cover many different
energy and pollution related attributes. The two attributes relevant to this study are
Site EUI (kBtu/ft2) and Direct GHG Emissions (MtCO2e). Site Energy Use Intensity is the
amount of energy consumed by the property in British thermal units per gross square foot.
Direct GHG Emissions are the total greenhouse gases emitted by the property in metric
tons of carbon dioxide equivalent.

Site EUI is used to understand the energy intensity of a building, while Direct Emissions
will be used to determine if there is a relationship between greenhouse gas emitting
buildings and asthma rates within an area. The data collected was for the year 2013.
This year was chosen to fit in the middle of the three-year average (2012-2014) collected for
asthma hospitilizaion.

The dataset comes as a list of individual buildings and their performance in many energy
related metrics. The buildings were grouped by zip code and their metrics summed to a list
of 181 zip codes with the total Site EUIs and Direct Emissions.

\subsection{Neighborhood Income and Population}
Neighborhood and Income data were collected through the American Census Bureau's website.
The data comes from the American Community Survey 5-year average over the years 2011-2015.
These years were selected to overlap with the asthma hospitalization rates as much as possible.
The attributes collected include household median income, income per capita, and total
population per zip code tabulation area.

\subsection{Merging}
The dataset was merged by zip code. As much as zip codes are not the ideal demographic
tabulation unit, they were necessary in this analysis as they provided the smallest level
of resolution available for asthma hosptilization rates. The American Community Survey
has its own proxy for zip code knows as a Zip Code Tabulation Area. ZCTAs allow for easy
merging with other zip code indexed data. After merging all datasets, and dropping
any zip codes that did not have all attributes available, primarily because of low number
of asthma cases, 164 zip codes remained from the original 181.

\subsection{Mapping}
An important way to understand data of this kind is to see it on a map.
To map the data, a shapefile of zip code boundaries was downloaded from NYC Open Data.

\section{Results}
The two primary questions this project aims to answer is is what, if any, relationship there is between
building energy use and direct emissions to local asthma hospitilization rates. These results
will be compared to how income affects energy use and asthma hospitilization rates.

\subsection{Visualization}

When comparing the maps of energy use and emissions, it is also not surprising to see
similar patterns emerge as well. It makes sense to see Manhattan being the most energy
intensive, but it is somewhat surprising to see that Manhattan also has some of the highest
concentration of direct emissions.

% \begin{figure}[!t]
% \centering
% \includegraphics[width=3.5in]{MapAsthma}
% \caption{Asthma Hospitilization Rate per 10,000 residents}
% \label{fig_map_asthma}
% \end{figure}

\subsection{Outliers}

\subsection{Asthma Discharge Rate}


\subsection{Building Energy Use}

\section{Discussion and Conclusion}
The results above are not what one might intuitively expect. There is essentially
no relationship between energy use of buildings or their direct emissions of greenhouse
gases on the asthma hospitilization rates of the zip code in which they are located.
Instead, a much better indicator of the severe asthma rate in a region is the median
household income. This might lead one to believe that the healthcare and living conditions
provided by higher wealth have a much larger impact in New York City on developing
and treating asthma than environmental factors. This analysis is not comprehensive
enough to make that kind of conclusion, though, as there very well may be other environmental
factors that are not taken into account by the building energy dataset that cause
asthma. It may also be that greater wealth clusters people in regions with lower environmental
contagions, which result in the relationship between wealth and asthma. Futher analysis is
needed.

It does appear that wealthier neighborhoods use less energy than those that are less
wealthy, though this relationship is not particularly strong. This may be
the result of new buildings built with higher efficiency HVAC systems and better
insulation. Regardless, the trend is not very strong and would require further
investigation to understand how higher wealth may result in lower building energy use.

As it stands, the strongest insight provided by this analysis is that the wealthier
a neighborhood, the lower the number of asthma-related hospitilizations take place.

%
%\begin{figure*}[!t]
%\centering
%\subfloat[Case I]{\includegraphics[width=2.5in]{box}%
%\label{fig_first_case}}
%\hfil
%\subfloat[Case II]{\includegraphics[width=2.5in]{box}%
%\label{fig_second_case}}
%\caption{Simulation results for the network.}
%\label{fig_sim}
%\end{figure*}
%
% Note that often IEEE papers with subfigures do not employ subfigure
% captions (using the optional argument to \subfloat[]), but instead will
% reference/describe all of them (a), (b), etc., within the main caption.
% Be aware that for subfig.sty to generate the (a), (b), etc., subfigure
% labels, the optional argument to \subfloat must be present. If a
% subcaption is not desired, just leave its contents blank,
% e.g., \subfloat[].


% An example of a floating table. Note that, for IEEE style tables, the
% \caption command should come BEFORE the table and, given that table
% captions serve much like titles, are usually capitalized except for words
% such as a, an, and, as, at, but, by, for, in, nor, of, on, or, the, to
% and up, which are usually not capitalized unless they are the first or
% last word of the caption. Table text will default to \footnotesize as
% the IEEE normally uses this smaller font for tables.
% The \label must come after \caption as always.
%
%\begin{table}[!t]
%% increase table row spacing, adjust to taste
%\renewcommand{\arraystretch}{1.3}
% if using array.sty, it might be a good idea to tweak the value of
% \extrarowheight as needed to properly center the text within the cells
%\caption{An Example of a Table}
%\label{table_example}
%\centering
%% Some packages, such as MDW tools, offer better commands for making tables
%% than the plain LaTeX2e tabular which is used here.
%\begin{tabular}{|c||c|}
%\hline
%One & Two\\
%\hline
%Three & Four\\
%\hline
%\end{tabular}
%\end{table}

% trigger a \newpage just before the given reference
% number - used to balance the columns on the last page
% adjust value as needed - may need to be readjusted if
% the document is modified later
%\IEEEtriggeratref{8}
% The "triggered" command can be changed if desired:
%\IEEEtriggercmd{\enlargethispage{-5in}}

% references section

% can use a bibliography generated by BibTeX as a .bbl file
% BibTeX documentation can be easily obtained at:
% http://mirror.ctan.org/biblio/bibtex/contrib/doc/
% The IEEEtran BibTeX style support page is at:
% http://www.michaelshell.org/tex/ieeetran/bibtex/
% argument is your BibTeX string definitions and bibliography database(s)

\bibliographystyle{IEEEtran}
\bibliography{library}

% that's all folks
\end{document}
